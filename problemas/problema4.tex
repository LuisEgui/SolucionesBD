Con el fin de abreviar el modelo, definimos:
\begin{center}
	\begin{tabular}{l}
		$S \equiv Sede(\underline{NombComp}, \underline{Ciudad})$\\
		$T \equiv Trabaja(\underline{NombEmp}, \underline{NombComp}, sueldo)$\\
		$V \equiv Vive(\underline{NombEmp}, Calle, Ciudad)$\\
		$J \equiv Jefes(\underline{NombEmp}, \underline{NombJefe})$\\
	\end{tabular}
\end{center}
\begin{enumerate}
	% a)
	\item Por simplificar, definimos:
	$$\rho_{NombEmp\rightarrow neT} (T) \land \rho_{NombComp\rightarrow ncT} (T)$$
	Luego,
	$$\alpha \leftarrow \Pi_{neT}(\sigma_{ncT = ``IBM"} (T))$$
	
	% b)
	\item
	$$\Pi_{neT}(T)-\alpha$$
	
	% c)	
	\item
	$$\Pi_{neT}(\sigma_{ncT = ``IBM" \land sueldo > 2000}(T))$$
	
	% d)
	\item Definimos:
	$$\rho_{ciudad \rightarrow cs} (S) \land \rho_{ciudad \rightarrow cv}(V) \land \rho_{NombEmp\rightarrow neV} (V) \land \rho_{NombComp\rightarrow ncS} (S)$$
	Luego,
	$$\sigma_{cs = cv}(S\innerjoin[ncS = ncT] (T\innerjoin[neT = neV] V))$$
	
	% e)
	\item Se entiende que $nJ$ es $FK$ de $V$.\\ 
	Definimos:
	$$\rho_{calle\rightarrow c}(V) \land \rho_{NombEmp\rightarrow neJ}(J) \land \rho_{NombJefe \rightarrow nJ}(J)$$
	Luego, obtenemos la direccion de los jefes:
	$$\gamma \leftarrow \amalg_{nJ}(V \innerjoin[neV = nJ] J)$$
	La relacion $\gamma = (\underline{neJ}, \underline{c}, \underline{cv})$ tal que $c$ y $cv$ son las calles y ciudades - respectivamente - de los jefes de $neJ$.\\
	La operacion que nos hace obtener aquellos empleados cuyas direcciones concuerdan con \textbf{cada una} de las direcciones de sus jefes es:
	$$\Pi_{neV} (V \div \gamma)$$
	No obstante dado que un $``$Empleado$"$  puede tener varios domicilios y sabiendo que $nJ \in V$, la operacion anterior no nos es del todo valida porque en la consulta se pide por la misma direccion que la de $``$su$"$ jefe - sin indicar que jefe.\\
	En caso de especificar en la consulta que la direccion del empleado sea la misma que la de \textbf{al menos uno} de sus jefes, la consulta seria la siguiente:
	$$\Pi_{neV}(V\cap \gamma)$$
	
	% f)
	\item Definimos, con el fin de sintetizar posteriormente;
	$$\delta \leftarrow (\sigma_{ncS = ``IBM"} (S))$$
	Obtenemos las ciudades \textit{sede} de $``$IBM$"$:
	$$\lambda \leftarrow \Pi_{cs}(\delta)$$
	Asi pues, las empresas que tienen sede en cada una de las ciudades en las que tiene sede IBM son:
	$$\theta \leftarrow S \div \lambda$$
	Sin embargo, en la consulta anterior sale incluida la propia IBM. Por tanto, realizamos lo siguiente:
	$$\theta - \Pi_{ncS}(\delta)$$
	
\end{enumerate}