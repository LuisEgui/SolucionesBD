Con el fin de abreviar el modelo, definimos:
\begin{center}
	\begin{tabular}{l}
		$C \equiv Clientes(\underline{NCliente}, Nombre, Direccion, Telefono, Poblacion)$\\
		$P \equiv Producto(\underline{CodProducto}, Descripcion, Precio)$\\
		$V \equiv Venta(\underline{idVenta}, CodProducto, NCliente, Cantidad)$\\
	\end{tabular}
\end{center}
\begin{enumerate}
	% a)
	\item 
	$$\Pi_{nombre}(\sigma_{dirrecion = Palencia}(C))$$
	
	% b)
	\item Esta consulta no me queda totalmente clara. Entiendo que se pide que \underline{CodProducto} = Descripcion (P); no obstante, es extraño ya que $type[CodProducto]$ puede ser distinto de $type[Descripcion]$.
	$$\rho_{CodProducto\rightarrow cp}(P) \land \rho_{descripcion \rightarrow desc}(P)$$
	$$\sigma{cp = desc}(P)$$
	
	% c)
	\item Definimos:
	$$\rho_{NCliente \rightarrow nc} (C) \land \rho_{NCliente \rightarrow ncV} (V)$$
	Luego,
	$$\amalg _{cp}(C\innerjoin[ncC = ncV] (\sigma_{cantidad > 500} (V)))$$
	
	% d)
	\item 
	$$\Pi_{nombre}(C - (C\innerjoin[ncC = ncV] V)) $$
	
	% e)
	\item 
	$$\Pi_{cp}(V\innerjoin[ncV = ncC] (\sigma_{direccion = ``Palencia" \vee direccion = ``Valladolid"} (C)))$$
	
\end{enumerate}