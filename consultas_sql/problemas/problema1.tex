Dado el modelo relacional:
\begin{center}
	\begin{tabular}{l}
		$Pilotos(\underline{iniciales}, nombre, escuderia, pais)$\\
		$Circuitos(\underline{nombre}, longitud, nVueltas)$\\
		$Parrilla(\underline{piloto}, \underline{circuito}, posicion)$\\				$Tiempos(\underline{piloto}, \underline{circuito}, \underline{nVuelta}, tiempo, paradaBoxes)$\\
	\end{tabular}
\end{center}

\begin{enumerate}

	% a)
	\item 
	Realizando la consulta con un producto cartesiano, seria:
	\begin{lstlisting}[language=sql]
	select p1.*
	from Pilotos as p1, Pilotos as p2
	where p1.iniciales != p2.iniciales -- distinct
	and p1.pais = p2.pais and p1.escuderia = p2.escuderia;\end{lstlisting}
	
	Haciendo uso del \textit{join}, seria:
	\begin{lstlisting}[language=sql]
	select p1.*
	from Pilotos p1 
	     join Pilotos p2 on p1.escuderia = p2.escuderia
	where p1.pais = p2.pais and p1.iniciales != p2.iniciales;\end{lstlisting}
	
	% b)
	\item 
	\begin{lstlisting}[language=sql]
	select distinct escuderia
	from Pilotos 
	     inner join Parrilla on iniciales = piloto
	where posicion in ('1','2','3');\end{lstlisting}
	
	% c)
	\item 
	\begin{lstlisting}[language=sql]
	select circuito, max(tiempo) as tiempo_max_vuelta
	from Tiempos
	group by circuito;\end{lstlisting}

	\newpage
	% d)
	\item 
	\begin{lstlisting}[language=sql]
	select circuito, piloto, count(nVuelta) as vueltas_completadas, 
	       sum(tiempo) as tiempo_total
	from Tiempos
	where circuito = 'Monza'
	group by circuito, piloto
	order by vueltas_completadas desc, tiempo_total asc;\end{lstlisting}
	
	% e)
	\item 
	Haciendo uso de un SQL mas estandar:
	\begin{lstlisting}[language=sql]
	select nombre, longitud
	from Circuitos
	where longitud = (
			select max(longitud)
			from Circuitos
			);\end{lstlisting}
	
	Para OracleSQL 12c R1 (12.1):
	\begin{lstlisting}[language=sql]
	select nombre, longitud
	from Circuitos
	order by longitud desc
	fetch first row with ties;\end{lstlisting}
	
	% f)
	\item
	\begin{lstlisting}[language=sql]
	select piloto, circuito, sum(tiempo) as tiempo_invertido, 
		   count(nVuelta) as vueltas_totales
	from Tiempos
		 inner join Circuitos on circuito = nombre
	group by piloto, circuito
	having count(vueltas_totales) = nVueltas;\end{lstlisting}
	
	\newpage
	
	% g)
	\item
	Suponemos que \textit{Tiempos(paradaBoxes)} se trata de un \textit{boolean} cuyo valor
	es \textit{1} cuando el piloto ha realizado la parada en la vuelta indicada en el
	registro; y que por defecto su valor es \textit{0}.
	\begin{lstlisting}[language=sql]
	select circuito, avg(paradaBoxes) as media_paradas_boxes
	from Tiempos
	group by circuito;\end{lstlisting}
\end{enumerate}