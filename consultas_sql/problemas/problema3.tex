Dado el modelo relacional:
\begin{center}
	\begin{tabular}{l}
		$Montura(\underline{codigoMontura}, nombre, sexo, edad)$\\
		$Monta(\underline{dni}, \underline{codigoMontura}, \underline{idCompeticion}, clasificacion)$\\
		$Jinete(\underline{dni}, nombre, apellidos, categoria, sexo, cache)$\\
		$Competicion(\underline{id}, nombre, fecha, lugar)$\\	
	\end{tabular}
\end{center}

\begin{enumerate}
	
	% a)
	\item \begin{lstlisting}[language=sql]
	select count(*) as num_competiciones
	from Competicion
	where fecha = 2014 and lugar = `Madrid';\end{lstlisting}
	
	% b)
	\item \begin{lstlisting}[language=sql]
	select sexo, categoria, avg(cache) as cache_medio
	from Jinete
	where sexo = 'Mujer'
	group by sexo, categoria;\end{lstlisting}
	
	% c()
	\item \begin{lstlisting}[language=sql]
	select idCompeticion, nombre, count(distinct dni) as num_participantes
	from Monta
		 inner join Competicion on id = idCompeticion
	group by idCompeticion, nombre
	order by num_participantes desc
	fetch first row with ties;\end{lstlisting}
	
	% d)
	\item \begin{lstlisting}[language=sql]
	select nombre, apellidos, categoria, 
		   count(distinct idCompeticion) as num_participaciones
	from Monta
		 inner join Jinete on Monta.dni = Jinete.dni
	where sexo = 'Hombre' and clasificacion > 1
	group by nombre, apellidos, categoria
	having count(num_participaciones) >= 100
	order by num_participaciones desc;\end{lstlisting}
	
	% e)
	\item \begin{lstlisting}[language=sql]
	select nombre, apellidos, categoria, cache
	from Jinete, (	select categoria, avg(cache) as media_cache
					from Jinete
					where sexo = 'Hombre'
					group by categoria
					) referencia_masculina
	where sexo = 'Mujer' 
		  and Jinete.categoria = referencia_masculina.categoria
		  and cache > referencia_masculina.media_cache;\end{lstlisting}
		  
	% f)
	\item \begin{lstlisting}[language=sql]
	select avg(cache)
	from (
		select distinct dni, cache
		from Monta
			 inner join Jinete on Monta.dni = Jinete.dni
		where clasificacion in ('1','2','3')
	);\end{lstlisting}
\end{enumerate}