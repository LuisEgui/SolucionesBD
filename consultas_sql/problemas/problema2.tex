Dado el modelo relacional:
\begin{center}
	\begin{tabular}{l}
		$Persona(\underline{dni}, sexo, edad)$\\
		$Habla(\underline{dniPersona}, \underline{idioma}, grado)$\\
		$SolicitaCita(\underline{dniSolicitante}, \underline{dniSolicitado}, idioma)$\\	
	\end{tabular}
\end{center}

\begin{enumerate}
	
	% a)
	\item 
	\begin{lstlisting}[language=sql]
	select idioma, count(*) as no_hablantes
	from Habla
	where grado = 3
	group by idioma
	order by no_hablantes desc
	fetch first row with ties;\end{lstlisting}
	
	% b)
	\item 
	\begin{lstlisting}[language=sql]
	select avg(edad) as promedio_edad_hispanohablantes
	from Habla h1
		 inner join Persona on h1.dniPersona = dni
	where idioma = 'Espanol' and not exists (
			select idioma
			from Habla h2
			where h1.dniPersona = h2.dniPersona and h2.idioma != `Espanol');\end{lstlisting}
			
	% c)
	\item 
	Esta consulta no nos devuelve una unica fila, sino que para cada persona; nos devuelve el numero de idiomas que habla. AL estar ordenadas de mayor numero de idiomas hablados, a menor: obtendremos en primera posicion la persona que mas idiomas habla.
	\begin{lstlisting}[language=sql]
	select dni, sexo, edad, count(idioma) as lenguas_habladas
	from Habla
		 inner join Persona on dniPersona = dni
	group by dni
	order by lenguas_habladas desc;
	-- lenguas_habladas son filas unicas ya que idioma es unique.\end{lstlisting}
	
	\newpage
	En MySQL y PostgreSQL es sencillo poder limitar el numero de filas obtenidas en la anterior consulta, de la siguiente manera:
	\begin{lstlisting}[language=sql]
	select dni, sexo, edad, count(idioma) as lenguas_habladas
	from Habla
		 inner join Persona on dniPersona = dni
	group by dni
	order by lenguas_habladas desc
	limit 1;\end{lstlisting}
	
	Para Oracle SQL 12c R1 (12.1):
	\begin{lstlisting}[language=sql]
	select sexo, edad
	from (
		  select dni, sexo, edad, count(idioma) as lenguas_habladas
		  from Habla
		  	   inner join Persona on dniPersona = dni
		  group by dni, sexo, edad
		  order by lenguas_habladas desc
		  fetch first row with ties	
	);\end{lstlisting}

	% d)
	\item De manera trivial:
	\begin{lstlisting}[language=sql]
	select p_req.dni, count(*) as num_citas
	from SolicitaCita
		 inner join Persona p on dniSolicitante = p.dni
		 inner join Persona p_req on dniSolicitado = p_req.dni
	where p.sexo != p_req.sexo
	group by p_req.dni
	having count(*) = (
				select max(num_citas)
				from (
					   select p2.dni, count(*) as num_citas
					   from SolicitaCita
					   		inner join Persona p1 on dniSolicitante = p1.dni
					   		inner join Persona p2 on dniSolicitado = p2.dni
					   	where p1.sexo != p2.sexo
					   	group by p2.dni
					 )
				);\end{lstlisting}
	\newpage
	Para Oracle 12c R1 (12.1):
	\begin{lstlisting}[language=sql]
	select edad, sexo
	from (
			select p2.dni, p2.edad, p2.sexo, count(*) as num_citas
			from SolicitaCita
				 inner join Persona p1 on dniSolicitante = p1.dni
	 			 inner join Persona p2 on dniSolicitado = p2.dni
			where p1.sexo != p2.sexo
			group by p2.dni, p2.edad, p2.sexo
			order by num_citas desc
			fetch first row with ties
		 );\end{lstlisting}
	
	% e)
	\item Recordamos que piden los idiomas hablados por el solicitante/solicitado pero que no han sido pedidos en la cita. Por lo que la consulta seria:
	\begin{lstlisting}[language=sql]
	select distinct h1.idioma -- idiomas que habla el solicitante
	from SolicitaCita
		 inner join Habla h1 on dniSolicitante = dniPersona
	union -- une filas de manera vertical, sin duplicarlas.
	select distinct h2.idioma -- idiomas que habla el solicitado
	from SolicitaCita
		 inner join Habla h2 on dniSolicitado = dniPersona
	minus
	select distinct idioma -- idiomas pedidos en las citas
	from SolicitaCita;\end{lstlisting}
	
	Tambien es posible realizar la consulta haciendo uso del \textit{left join} sobre la tabla $SolicitaCita$.
	En cuanto a cual seria la consulta mas optima, seria necesario realizar una prueba dado que, aunque la operacion \textit{minus} es pobre en optimizacion, esta se realiza sobre valores ya filtrados anteriormente.
	
	% f)
	\item 
	\begin{lstlisting}[language=sql]
	select distinct h1.dniPersona, h1.grado
	from SolicitaCita
		 inner join Habla h1 on dniSolicitante = h1.dniPersona
		 inner join Habla h2 on dniSolicitado = h2.dniPersona
		 inner join Persona on dniSolicitado = dni
	where sexo = 'M' and edad < 25 
		  and h1.idioma = 'Ingles' and h2.idioma = `Ingles';\end{lstlisting}
	
	% g)
	\item
	\begin{lstlisting}[language=sql]
	select dni_no_solicitado, sexo, edad, 
		   count(idioma) as num_idiomas_hablados
	from ( 
		  select dni as dni_no_solicitado
		  from Persona
		  minus -- personas nunca solicitadas
		  select distinct dniSolicitado
		  from SolicitaCita
		 )
		 -- puede que exista la persona pero no hable ningun idioma
		 left join Habla on dni_no_solicitado = dniPersona
		 inner join Persona p on dni_no_solicitado = p.dni
	group by dni_no_solicitado, sexo, edad
	order by num_idiomas_hablados desc;\end{lstlisting}
\end{enumerate}