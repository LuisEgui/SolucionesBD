Dado el modelo relacional:
\begin{center}
	\begin{tabular}{l}
		$Competicion(\underline{id}, nombre, fecha, lugar, ambito)$\\
		$Participa(\underline{dni}, \underline{idComp}, clasificacion)$\\
		$Patinador(\underline{dni}, nombre, apellidos, sexo, fechaNac, pais)$\\
	\end{tabular}
\end{center}

\begin{enumerate}
	% a)
	\item \begin{lstlisting}[language=sql]
	select count(*) as num_comp
	from Participa
		 inner join Competicion on id = idComp
	where lugar = 'Chicago'
		  and (fecha >= 2010 and fecha <= 2015)
		  and clasificacion is not null; -- si !null => se ha celebrado\end{lstlisting}
		  
	% b)
	\item \begin{lstlisting}[language=sql]
	select nombre, apellidos
	from Participa
		 inner join Patinador on Patinador.dni = Participa.dni
		 inner join Competicion on id = idComp
	where sexo = 'M' and ambito = 2 -- Ambito: (2) Camp. Mundial
		  and clasificacion = 1;\end{lstlisting}
		  
	% c)
	\item  \begin{lstlisting}[language=sql]
	select pais, avg(clasificacion) as media_clasif
	from Participa
		 inner join Patinador on Patinador.dni = Participa.dni
	where clasificacion is not null -- si !null => se ha celebrado
		  and sexo = 'H'
	group by pais
	order by media_clasif asc;\end{lstlisting}
	
	% d)
	\item Esta consulta no deja claro si la competicion se ha celebrado o no.\begin{lstlisting}[language=sql]
	select id, nombre, count(dni) as num_participantes
	from Participa
		 inner join Competicion on id = idComp
	group by id, nombre
	order by num_participantes desc
	fetch first row with ties;\end{lstlisting}
	
	Haciendo uso de un SQL mas estandar:
	
	\begin{lstlisting}[language=sql]
	select id, nombre, count(dni) as num_participantes
	from Participa
		 inner join Competicion on id = idComp
	group by id, nombre
	having count(dni) = (
			select max(num_participantes)
			from (
				 select idComp, count(dni) as num_participantes 
				 from Participa
				 group by idComp
			)
		);\end{lstlisting}
		
	% e)
	\item No me queda claro que es lo que pide exactamente.
\end{enumerate}