% -------------------------------------------------------------------------------------
% Proyecto: Resolucion de problemas del tema 6 de Bases de Datos: Consultas SQL
%
% Fecha: Noviembre 2019
%
% Este documento contiene únicamente las soluciones a las que he llegado para los
% problemas expuestos en la asignatura "Bases de Datos".
%
% Autor: Luis Egui.
% 
% -------------------------------------------------------------------------------------

\documentclass[a4paper, 11pt, twoside]{article}

% -------------------------------------------------------------------------------------
% Paquetes
% -------------------------------------------------------------------------------------

% Encabezado
\usepackage[headings]{fullpage}
\usepackage{fancyhdr}
% Referencias ult. pag
\usepackage{lastpage}
% Encoding y noto font
\usepackage{noto-serif}
\usepackage[T1]{fontenc}
% Leyenda/pie de foto (fig., imágenes...)
\usepackage[font=small, labelfont=bf]{caption}
\usepackage{subcaption}
\usepackage[ragged]{sidecap}
% Abultamiento de caracteres, expansión de la tipografía
\usepackage[protrusion=true, expansion=true]{microtype}
% Referencias cruzadas, etc...
\usepackage[linktocpage=true]{hyperref}
% Lenguaje en español (principal) e inglés
\usepackage[main=spanish, english]{babel}
% UTF-8 Encoding
\usepackage[utf8]{inputenc}
% Añade comandos para cambiar las fuentes de las secciones, encabezados, etc, que indiques.
\usepackage{sectsty}
% Define todas los símbolos encontrados en el AMS: msam y msbm. En resumen: símbolos matemáticos.
\usepackage{amssymb}
\usepackage{amsmath}
% Sangría
\usepackage{ragged2e}
% Tabla de contenidos
\usepackage{tocbasic}
\usepackage{xcolor}
% Espaciado
\usepackage{setspace}
% Tablas de "calidad"
\usepackage{booktabs}
% Para crear multiples columnas en una enumeracion
\usepackage{multicol}
\usepackage{enumitem}
% Modificar la apariencia del "título"
\usepackage{titling}
% Comillas
\usepackage{textcmds}
% Ref titulos secciones
\usepackage{nameref}
% Codigo SQL
\usepackage{listings}

% -------------------------------------------------------------------------------------
% Macro's
% -------------------------------------------------------------------------------------

\newcommand{\HRule}[1]{\rule{\linewidth}{#1}}
\onehalfspacing
\setcounter{tocdepth}{5}
\setcounter{secnumdepth}{5}

\newcommand*\tocrule[2][-1.5\dp\strutbox]{%
	\makebox[0pt][r]{\color{#2}\rule[#1]{\textwidth}{.4pt}}%
}

\renewcommand\labelenumi{\roman{enumi})}
\renewcommand\theenumi\labelenumi

% Inner Join
\newcommand{\innerjoin}[1][\ensuremath{\theta}]{\underset{#1}{\Join}}

% Enums empezando por a), b), ...
\renewcommand{\labelenumi}{\alph{enumi})}

% Highlight COLOR SQL
\definecolor{dkgreen}{rgb}{0,0.6,0}
\definecolor{ltgray}{rgb}{0.5,0.5,0.5}

% Highlights de SQL
\lstset{%
	backgroundcolor=\color{white},
	basicstyle=\footnotesize,
	breakatwhitespace=false,
	breaklines=true,
	captionpos=b,
	commentstyle=\color{dkgreen},
	deletekeywords={...},
	escapeinside={\%*}{*)},
	extendedchars=true,
	frame=,
	keepspaces=true,
	keywordstyle=\color{blue},
	language=SQL,
	morekeywords={*,modify,MODIFY,minus,fetch,...},
	numbers=left,
	numbersep=2pt,
	numberstyle=\tiny,
	rulecolor=\color{ltgray},
	showspaces=false,
	showstringspaces=false, 
	showtabs=false,
	stepnumber=1,
	tabsize=4,
	title=\lstname
}

% -------------------------------------------------------------------------------------
% Metadatos
% -------------------------------------------------------------------------------------

\hypersetup{
	pdftitle={Resolución_de_problemas_de_Álgebra_Relacional},
	pdfauthor={Luis Egui},
	pdfkeywords={resolucion, algebrarelacional, database, ejercicios},
	pdfnewwindow=true,
	colorlinks=false,
	linktoc=page
}

% -------------------------------------------------------------------------------------
% Encabezado
% -------------------------------------------------------------------------------------

\pagestyle{fancy}
\fancyhf{}
\setlength\headheight{15pt}
\fancyhead[L]{Soluciones}
\fancyhead[R]{Álgebra Relacional}
\fancyfoot[R]{[\thepage/\pageref{LastPage}]}


% -------------------------------------------------------------------------------------
% Portada
% -------------------------------------------------------------------------------------

\title{ 
	\HRule{0.5pt} \\
	\Large \textbf{\uppercase{Resolución de consultas \\SQL}}
	\HRule{2pt} \\ [0.5cm]
	\normalsize \textsc{Pendiente de revisión} \vspace*{5\baselineskip}
}

\date{}

\author{
	Luis Egui \\ 
	\hspace{1.5mm}\\
}

% -------------------------------------------------------------------------------------
% Estilo de las secciones
% -------------------------------------------------------------------------------------

\sectionfont{\fontfamily{noto}\normalfont}

% Framing section headings
\makeatletter
\def\section{\@ifstar\unnumberedsection\numberedsection}
\def\numberedsection{\@ifnextchar[%]
	\numberedsectionwithtwoarguments\numberedsectionwithoneargument}
\def\unnumberedsection{\@ifnextchar[%]
	\unnumberedsectionwithtwoarguments\unnumberedsectionwithoneargument}
\def\numberedsectionwithoneargument#1{\numberedsectionwithtwoarguments[#1]{#1}}
\def\unnumberedsectionwithoneargument#1{\unnumberedsectionwithtwoarguments[#1]{#1}}
\def\numberedsectionwithtwoarguments[#1]#2{%
	\ifhmode\par\fi
	\removelastskip
	\vskip 3ex\goodbreak
	\refstepcounter{section}%
	\hbox to \hsize{%
		\fbox{%
			\hbox to 1cm{\hss\bfseries\Large\thesection.\ }%
			\vtop{%
				\advance \hsize by -1cm
				\advance \hsize by -2\fboxrule
				\advance \hsize by -2\fboxsep
				\parindent=0pt
				\leavevmode\raggedright\bfseries\Large
				#2
			}%
	}}\nobreak
	\vskip 2mm\nobreak
	%\addcontentsline{toc}{section}{%
	%	\protect\numberline{\thesection}%
	%	#1}%
	\ignorespaces
}
\def\unnumberedsectionwithtwoarguments[#1]#2{%
	\ifhmode\par\fi
	\removelastskip
	\vskip 3ex\goodbreak
	%  \refstepcounter{section}%
	\hbox to \hsize{%
		\fbox{%
			%      \hbox to 1cm{\hss\bfseries\Large\thesection.\ }%
			\vtop{%
				%        \advance \hsize by -1cm
				\advance \hsize by -2\fboxrule
				\advance \hsize by -2\fboxsep
				\parindent=0pt
				\leavevmode\raggedright\bfseries\Large
				#2
			}%
	}}\nobreak
	\vskip 2mm\nobreak
	%\addcontentsline{toc}{section}{%
	%    \protect\numberline{\thesection}%
	%	#1}%
	\ignorespaces
}
\makeatother

% -------------------------------------------------------------------------------------
\begin{document}
	
	% Cambio del título de "Índice general" a simplemente "Índice"
	\renewcommand{\contentsname}{Contenidos}
	
	\begin{titlepage}
		\clearpage
		\maketitle
		\thispagestyle{empty}
		\setcounter{page}{0}
	\end{titlepage}
	
	\newpage
	
	% Se añade al estilo del toc:
	% -> Lineas horizontales
	% -> "Página" encima del numero de página del "chapter" o "sector"
	\addtocontents{toc}{
		~\hfill\textbf{Página}
		\tocrule{black}
		\tocrule[\dimexpr\ht\strutbox+2\dp\strutbox]{black}
		\par
		\protect\thispagestyle{empty}
	}
	
	\tableofcontents
	\setcounter{page}{0}
	\newpage

	\section*{Problema 1}
	\addcontentsline{toc}{section}{Problema 1}\
	\begin{enumerate}
	% a)
	\item \begin{displaymath}
	\begin{split}
		R\cup S & = \{ (a, b), (b, c), (d, e)\} \cup \{ (b, c), (e, a), (b, d)\}\\
		& = \{ (a, b), (b, c), (d, e), (e, a), (b, d)\}
	\end{split}
	\end{displaymath}
	
	% b)
	\item \begin{displaymath}
		R-U = \{(a, b), (d, e)\}
	\end{displaymath}
	
	% c)
	\item \begin{displaymath}
	\begin{split}
		R\times U = \{ &(a, b, b, c), (a, b, e, a), (a, b, b, d),\\
		& (b, c, b, c), (b, c, e, a), (b, c, b, d),\\
		& (d, e, b, c), (d, e, e, a), (d, e, b, d)\}
	\end{split}
	\end{displaymath}
	
	% d)
	\item \begin{displaymath}
	\begin{split}
		\sigma_{A=C} (R\times U) & = \{(\alpha , \beta , \gamma, \delta) | \forall \alpha , \beta , \gamma, \delta \in (R\times U) \implies \alpha = \delta \}\\
		& = \{ (a, b, e, a), (d, e, b, d)\}
	\end{split}
	\end{displaymath}
	
	% e)	
	\item
	\begin{equation*}
	\begin{split}
		S\div T & = \{ b  |  \forall b \in S \land \forall c \in T \implies (b , c) \in S\}\\
		& = \{b\}
	\end{split}
	\end{equation*}
	
\end{enumerate}
	
	\newpage
	
	\section*{Problema 2}
	\addcontentsline{toc}{section}{Problema 2}\
	Con el fin de abreviar el modelo, definimos:
	\begin{center}
		\begin{tabular}{l}
			$P \equiv Proveedores(\underline{idp}, nombreP, categoria, ciudad)$\\
			$C \equiv Componentes(\underline{idc}, nombreC, color, peso, ciudad)$\\
			$A \equiv Articulos(\underline{ida}, nombreA, ciudad)$\\
			$E \equiv Envios(\underline{idp}, \underline{idc}, \underline{ida}, cantidad)$\\
		\end{tabular}
	\end{center}
	\begin{enumerate}
		\item $\Pi_{idp}(\sigma_{idc = C1 \wedge ida = A1}(E))$ % a)
		\item $\Pi_{ida}(\sigma_{idp = P1}(E))$ % b)
		\item Definimos:\\ % c)
		$\rho_{ciudad\rightarrow c_1}(A)$\\
		$\Pi_{idp}(\sigma_{color=``rojo"}(C\innerjoin[C.idc = E.idc](\sigma_{c_1 = ``Segovia" \vee c_1 = ``Barcelona"}(E \innerjoin[E.ida = A.ida] A))))$\\[0.25cm]
		Recordemos que $w_1 \innerjoin w_2 \equiv w_2 \innerjoin w_1$ , por lo que realmente da igual filtrar primero por %
		componentes \textit{rojos} enviados; que por artículos enviados fabricados en \textit{Segovia} o \textit{Barcelona}.
		\item Definimos:\\ % d)
		$\rho_{ciudad\rightarrow c_2}(P)$\\
		$\Pi_{idc}(\sigma_{c_2 = ``Segovia"} (P \innerjoin[P.idp = E.idp] (\sigma_{c_1 = ``Segovia"} (E \innerjoin[E.ida = A.ida] A))))$
		\item $\Pi_{color}(C\innerjoin[C.idc = E.idc](\sigma_{ipd = P1}(E)))$ % e)
		\item Definimos:\\ % f)
		$\rho_{ciudad\rightarrow c_3}(C)$\\
		$E\innerjoin[E.ida = A.ida] (A\innerjoin[c_1 = c_2](P\innerjoin[c_2 = c_2]C))$\\
	\end{enumerate}
	
	\newpage
	
\end{document}