% -------------------------------------------------------------------------------------
% Proyecto: Resolucion de problemas del tema 6 de Bases de Datos: Consultas SQL
%
% Fecha: Noviembre 2019
%
% Este documento contiene únicamente las soluciones a las que he llegado para los
% problemas expuestos en la asignatura "Bases de Datos".
%
% Autor: Luis Egui.
% 
% -------------------------------------------------------------------------------------

\documentclass[a4paper, 11pt, twoside]{article}

% -------------------------------------------------------------------------------------
% Paquetes
% -------------------------------------------------------------------------------------

% Encabezado
\usepackage[headings]{fullpage}
\usepackage{fancyhdr}
% Referencias ult. pag
\usepackage{lastpage}
% Encoding y noto font
\usepackage{noto-serif}
\usepackage[T1]{fontenc}
% Leyenda/pie de foto (fig., imágenes...)
\usepackage[font=small, labelfont=bf]{caption}
\usepackage{subcaption}
\usepackage[ragged]{sidecap}
% Abultamiento de caracteres, expansión de la tipografía
\usepackage[protrusion=true, expansion=true]{microtype}
% Referencias cruzadas, etc...
\usepackage[linktocpage=true]{hyperref}
% Lenguaje en español (principal) e inglés
\usepackage[main=spanish, english]{babel}
% UTF-8 Encoding
\usepackage[utf8]{inputenc}
% Añade comandos para cambiar las fuentes de las secciones, encabezados, etc, que indiques.
\usepackage{sectsty}
% Define todas los símbolos encontrados en el AMS: msam y msbm. En resumen: símbolos matemáticos.
\usepackage{amssymb}
\usepackage{amsmath}
% Sangría
\usepackage{ragged2e}
% Tabla de contenidos
\usepackage{tocbasic}
\usepackage{xcolor}
% Espaciado
\usepackage{setspace}
% Tablas de "calidad"
\usepackage{booktabs}
% Para crear multiples columnas en una enumeracion
\usepackage{multicol}
\usepackage{enumitem}
% Modificar la apariencia del "título"
\usepackage{titling}
% Comillas
\usepackage{textcmds}
% Ref titulos secciones
\usepackage{nameref}
% Codigo SQL
\usepackage{listings}

% -------------------------------------------------------------------------------------
% Macro's
% -------------------------------------------------------------------------------------

\newcommand{\HRule}[1]{\rule{\linewidth}{#1}}
\onehalfspacing
\setcounter{tocdepth}{5}
\setcounter{secnumdepth}{5}

\newcommand*\tocrule[2][-1.5\dp\strutbox]{%
	\makebox[0pt][r]{\color{#2}\rule[#1]{\textwidth}{.4pt}}%
}

\renewcommand\labelenumi{\roman{enumi})}
\renewcommand\theenumi\labelenumi

% Inner Join
\newcommand{\innerjoin}[1][\ensuremath{\theta}]{\underset{#1}{\Join}}

% Enums empezando por a), b), ...
\renewcommand{\labelenumi}{\alph{enumi})}

% Highlight COLOR SQL
\definecolor{dkgreen}{rgb}{0,0.6,0}
\definecolor{ltgray}{rgb}{0.5,0.5,0.5}

% Highlights de SQL
\lstset{%
	backgroundcolor=\color{white},
	basicstyle=\footnotesize,
	breakatwhitespace=false,
	breaklines=true,
	captionpos=b,
	commentstyle=\color{dkgreen},
	deletekeywords={...},
	escapeinside={\%*}{*)},
	extendedchars=true,
	frame=,
	keepspaces=true,
	keywordstyle=\color{blue},
	language=SQL,
	morekeywords={*,modify,MODIFY,minus,fetch,...},
	numbers=left,
	numbersep=2pt,
	numberstyle=\tiny,
	rulecolor=\color{ltgray},
	showspaces=false,
	showstringspaces=false, 
	showtabs=false,
	stepnumber=1,
	tabsize=4,
	title=\lstname
}

% -------------------------------------------------------------------------------------
% Metadatos
% -------------------------------------------------------------------------------------

\hypersetup{
	pdftitle={Resolución_de_consultas_sql},
	pdfauthor={Luis Egui},
	pdfkeywords={resolucion, sql, database, ejercicios},
	pdfnewwindow=true,
	colorlinks=false,
	linktoc=page
}

% -------------------------------------------------------------------------------------
% Encabezado
% -------------------------------------------------------------------------------------

\pagestyle{fancy}
\fancyhf{}
\setlength\headheight{15pt}
\fancyhead[L]{Soluciones}
\fancyhead[R]{Consultas SQL}
\fancyfoot[R]{[\thepage/\pageref{LastPage}]}


% -------------------------------------------------------------------------------------
% Portada
% -------------------------------------------------------------------------------------

\title{ 
	\HRule{0.5pt} \\
	\Large \textbf{\uppercase{Resolución de consultas \\SQL}}
	\HRule{2pt} \\ [0.5cm]
	\normalsize \textsc{Pendiente de revisión} \vspace*{5\baselineskip}
}

\date{}

\author{
	Luis Egui \\ 
	\hspace{1.5mm}\\
}

% -------------------------------------------------------------------------------------
% Estilo de las secciones
% -------------------------------------------------------------------------------------

\sectionfont{\fontfamily{noto}\normalfont}

% Framing section headings
\makeatletter
\def\section{\@ifstar\unnumberedsection\numberedsection}
\def\numberedsection{\@ifnextchar[%]
	\numberedsectionwithtwoarguments\numberedsectionwithoneargument}
\def\unnumberedsection{\@ifnextchar[%]
	\unnumberedsectionwithtwoarguments\unnumberedsectionwithoneargument}
\def\numberedsectionwithoneargument#1{\numberedsectionwithtwoarguments[#1]{#1}}
\def\unnumberedsectionwithoneargument#1{\unnumberedsectionwithtwoarguments[#1]{#1}}
\def\numberedsectionwithtwoarguments[#1]#2{%
	\ifhmode\par\fi
	\removelastskip
	\vskip 3ex\goodbreak
	\refstepcounter{section}%
	\hbox to \hsize{%
		\fbox{%
			\hbox to 1cm{\hss\bfseries\Large\thesection.\ }%
			\vtop{%
				\advance \hsize by -1cm
				\advance \hsize by -2\fboxrule
				\advance \hsize by -2\fboxsep
				\parindent=0pt
				\leavevmode\raggedright\bfseries\Large
				#2
			}%
	}}\nobreak
	\vskip 2mm\nobreak
	%\addcontentsline{toc}{section}{%
	%	\protect\numberline{\thesection}%
	%	#1}%
	\ignorespaces
}
\def\unnumberedsectionwithtwoarguments[#1]#2{%
	\ifhmode\par\fi
	\removelastskip
	\vskip 3ex\goodbreak
	%  \refstepcounter{section}%
	\hbox to \hsize{%
		\fbox{%
			%      \hbox to 1cm{\hss\bfseries\Large\thesection.\ }%
			\vtop{%
				%        \advance \hsize by -1cm
				\advance \hsize by -2\fboxrule
				\advance \hsize by -2\fboxsep
				\parindent=0pt
				\leavevmode\raggedright\bfseries\Large
				#2
			}%
	}}\nobreak
	\vskip 2mm\nobreak
	%\addcontentsline{toc}{section}{%
	%    \protect\numberline{\thesection}%
	%	#1}%
	\ignorespaces
}
\makeatother

% -------------------------------------------------------------------------------------
\begin{document}
	
	% Cambio del título de "Índice general" a simplemente "Índice"
	\renewcommand{\contentsname}{Contenidos}
	
	\begin{titlepage}
		\clearpage
		\maketitle
		\thispagestyle{empty}
		\setcounter{page}{0}
	\end{titlepage}
	
	\newpage
	
	% Se añade al estilo del toc:
	% -> Lineas horizontales
	% -> "Página" encima del numero de página del "chapter" o "sector"
	\addtocontents{toc}{
		~\hfill\textbf{Página}
		\tocrule{black}
		\tocrule[\dimexpr\ht\strutbox+2\dp\strutbox]{black}
		\par
		\protect\thispagestyle{empty}
	}
	
	\tableofcontents
	\setcounter{page}{0}
	\newpage
	
	% Cambio del titulo del abstracto:
	\renewcommand{\abstractname}{Estilo del codigo SQL usado}
	\setcounter{page}{0}
	
	% Abstracto: Estilo del codigo SQL usado
	\begin{abstract}
		\thispagestyle{empty}
		Para el codigo SQL me gusta que este sea lo mas legible posible, y, dado que SQL es un lenguaje declarativo de cuarta generacion (4GL)\footnote{\href{https://en.wikipedia.org/wiki/Fourth-generation_programming_language}{4th generation language}}, resulta mas comodo leerlo tal y como se lee un libro. De esta manera:\\
		
		\begin{itemize}
			% Sentencias, sintaxis
			\item La sintaxis SQL, independientemente del Database Management System (DBMS), esta escrita en \textit{lowercase}.
			
			% Atributos
			\item Los atributos indicados en el operador \textit{``select"} estan sujetos al nombre indicado en el modelo relacional dado. Los renombramientos de los mismos o de funciones de agregacion estan escritos en \textit{lower\_snake\_case}. \\
			Solo en caso necesario, se indicara: agregando el nombre de la tabla (o su correspondiente renombramiento) delante del atributo. 
			En mi opinion, aumenta la comprension de los atributos que resultan cabecera de la consulta que se esta haciendo.\\
			En caso de haber bastantes atributos, estos iran sangrados con respecto al operador \textit{``select"}. Por norma general, no suelo superar los 80 caracteres por linea.
			
			% Tablas
			\item Los nombres de las tablas estan escritos en \textit{CamelCase}, sin importar como hayan sido nombradas en el modelo.
			
			% Joins y subconsultas
			\item Los \textit{``joins"}, da igual el tipo, se realizaran con sangrado respecto del operador \textit{``from"} para asi tener clara la separacion de las distintas partes de la consulta y mejorar la comprension de la misma.\\
			Lo mismo es aplicable para subconsultas en el operador \textit{``from"} o en el operador \textit{``where"}.
		\end{itemize}
	\end{abstract}

	\newpage
	\section*{Problema 1}
	\addcontentsline{toc}{section}{Problema 1}\
	\begin{enumerate}
	% a)
	\item \begin{displaymath}
	\begin{split}
		R\bigcup S & = \{ (a, b), (b, c), (d, e)\} \bigcup \{ (b, c), (e, a), (b, d)\}\\
		& = \{ (a, b), (b, c), (d, e), (e, a), (b, d)\}
	\end{split}
	\end{displaymath}
	% b)
	\item \begin{displaymath}
		R-U = \{(a, b), (d, e)\}
	\end{displaymath}
	% c)
	\item \begin{displaymath}
	\begin{split}
		R\times U = \{ &(a, b, b, c), (a, b, e, a), (a, b, b, d),\\
		& (b, c, b, c), (b, c, e, a), (b, c, b, d),\\
		& (d, e, b, c), (d, e, e, a), (d, e, b, d)\}
	\end{split}
	\end{displaymath}
	% d)
	\item \begin{displaymath}
	\begin{split}
		\sigma_{A=C} (R\times U) & = \{(\alpha , \beta , \gamma, \delta) | \forall \alpha , \delta \implies \alpha = \delta \}\\
		& = \{ (a, b, e, a), (d, e, b, d)\}
	\end{split}
	\end{displaymath}
	% e)	
	\item Dado que no estoy completamente seguro de que la respuesta sea la correcta, añadiré las dos respuestas que creo que son válidas:
	% e) Respuesta 1
	\begin{equation}
	\begin{split}
		S\div T & = \{ \beta  |  \forall \beta , c_1 \in S \land c_2 \in T \implies c_1 = c_2 \}\\
		& = \{(b)\}
	\end{split}
	\end{equation}
	% e) Respuesta 2
	\begin{equation}
	\begin{split}
		S\div T & = \{ (\beta , \delta ) |  \forall \delta , c_1 \in S \land c_2 \in T \implies \delta = c_1 \land \delta = c_2 \}\\
		& = \{(b, c), (b, d)\}
	\end{split}
	\end{equation}
\end{enumerate}
	
	\newpage
	
	\section*{Problema 2}
	\addcontentsline{toc}{section}{Problema 2}\
	Con el fin de abreviar el modelo, definimos:
	\begin{center}
		\begin{tabular}{l}
			$P \equiv Proveedores(\underline{idp}, nombreP, categoria, ciudad)$\\
			$C \equiv Componentes(\underline{idc}, nombreC, color, peso, ciudad)$\\
			$A \equiv Articulos(\underline{ida}, nombreA, ciudad)$\\
			$E \equiv Envios(\underline{idp}, \underline{idc}, \underline{ida}, cantidad)$\\
		\end{tabular}
	\end{center}
	\begin{enumerate}
		\item $\Pi_{idp}(\sigma_{idc = C1 \wedge ida = A1}(E))$ % a)
		\item $\Pi_{ida}(\sigma_{idp = P1}(E))$ % b)
		\item Definimos:\\ % c)
		$\rho_{ciudad\rightarrow c_1}(A)$\\
		$\Pi_{idp}(\sigma_{color=``rojo"}(C\innerjoin[C.idc = E.idc](\sigma_{c_1 = ``Segovia" \vee c_1 = ``Barcelona"}(E \innerjoin[E.ida = A.ida] A))))$\\[0.25cm]
		Recordemos que $w_1 \innerjoin w_2 \equiv w_2 \innerjoin w_1$ , por lo que realmente da igual filtrar primero por %
		componentes \textit{rojos} enviados; que por artículos enviados fabricados en \textit{Segovia} o \textit{Barcelona}.
		\item Definimos:\\ % d)
		$\rho_{ciudad\rightarrow c_2}(P)$\\
		$\Pi_{idc}(\sigma_{c_2 = ``Segovia"} (P \innerjoin[P.idp = E.idp] (\sigma_{c_1 = ``Segovia"} (E \innerjoin[E.ida = A.ida] A))))$
		\item $\Pi_{color}(C\innerjoin[C.idc = E.idc](\sigma_{ipd = P1}(E)))$ % e)
		\item Definimos:\\ % f)
		$\rho_{ciudad\rightarrow c_3}(C)$\\
		$E\innerjoin[E.ida = A.ida] (A\innerjoin[c_1 = c_2](P\innerjoin[c_2 = c_2]C))$\\
	\end{enumerate}
	
	\newpage
	
	\section*{Problema 3}
	\addcontentsline{toc}{section}{Problema 3}\
	Con el fin de abreviar el modelo, definimos:
\begin{center}
	\begin{tabular}{l}
		$C \equiv Programadores(\underline{DNI},Nombre, Direccion, Telefono)$\\
		$A \equiv Analistas(\underline{DNI}, Nombre, Direccion, Telefono)$\\
		$D \equiv Distribucion(\underline{CodigoProy}, \underline{DNIEmp}, Horas)$\\
		$P \equiv Proyectos(\underline{Codigo}, Descripcion, DniDir)$\\
	\end{tabular}
\end{center}

\begin{enumerate}
	
	% a)
	\item Definimos:
	$$\rho_{CodigoProy \rightarrow cp}(D) \land \rho_{DNIEmp \rightarrow dE}(D) \land \rho_{Codigo\rightarrow c}(P)$$
	Luego,
	$$\Pi_{cp}(\sigma_{dE = 4} (D))$$
	
	% b)
	\item 
	$$ E \leftarrow D \innerjoin[dE = DNI](C\cup A)$$
	
	% c)
	\item 
	$$\Pi_{Nombre}(A \innerjoin[DNI = dE](D\innerjoin[dE = DniDir] P))$$
	
	% d)
	\item 
	$$\amalg _{DniDir}(P \innerjoin[DniDir = DNI]E)$$
	
	% e)
	\item 
	$$ DE \leftarrow \Pi_{dE}(E)$$
	
	% f)
	\item 
	$$\Pi_{dE}(D\innerjoin[dE = DNI](C\cap A))$$
	
	% g)
	\item 
	$$\Pi_{DNI}(C\cup A) - DE$$
	
	% h)
	\item Definimos:
	$$ \alpha \leftarrow \Pi_{cp}(D\innerjoin[dE = DNI] A)$$
	Luego,
	$$\Pi_{cp}(D) - \alpha$$
	
	\newpage
	
	% i)
	\item Definimos:
	$$EA\leftarrow A - (C\cap A)$$
	Luego,
	$$DA\leftarrow P\innerjoin[DniDir = DNI]EA$$
	$$\Pi_{DNI}(DA)$$
	
	% j)
	\item Definimos:
	$$\rho_{Nombre\rightarrow nC}(C) \land \rho_{Descripcion \rightarrow desc}(P)$$
	Luego,
	$$ \beta \leftarrow \Pi_{cp, nC, horas}(D\innerjoin[dE = DNI]C)$$
	$$ \Pi_{desc, nC, horas}(P\innerjoin[c = cp]\beta)$$
	
	% k)
	\item Definimos:
	$$\rho_{Telefono\rightarrow tC}(C) \land \rho_{Telefono\rightarrow tA}(A)$$
	Luego,
	$$\Pi_{tC}(C\innerjoin[tC = tA] A)$$
\end{enumerate}
	
	\newpage
	
	\section*{Problema 4}
	\addcontentsline{toc}{section}{Problema 4}\
	Con el fin de abreviar el modelo, definimos:
\begin{center}
	\begin{tabular}{l}
		$S \equiv Sede(\underline{NombComp}, \underline{Ciudad})$\\
		$T \equiv Trabaja(\underline{NombEmp}, \underline{NombComp}, sueldo)$\\
		$V \equiv Vive(\underline{NombEmp}, Calle, Ciudad)$\\
		$J \equiv Jefes(\underline{NombEmp}, \underline{NombJefe})$\\
	\end{tabular}
\end{center}
\begin{enumerate}
	% a)
	\item Por simplificar, definimos:
	$$\rho_{NombEmp\rightarrow neT} (T) \land \rho_{NombComp\rightarrow ncT} (T)$$
	Luego,
	$$\alpha \leftarrow \Pi_{neT}(\sigma_{ncT = ``IBM"} (T))$$
	
	% b)
	\item
	$$\Pi_{neT}(T)-\alpha$$
	
	% c)	
	\item
	$$\Pi_{neT}(\sigma_{ncT = ``IBM" \land sueldo > 2000}(T))$$
	
	% d)
	\item Definimos:
	$$\rho_{ciudad \rightarrow cs} (S) \land \rho_{ciudad \rightarrow cv}(V) \land \rho_{NombEmp\rightarrow neV} (V) \land \rho_{NombComp\rightarrow ncS} (S)$$
	Luego,
	$$\sigma_{cs = cv}(S\innerjoin[ncS = ncT] (T\innerjoin[neT = neV] V))$$
	
	% e)
	\item Se entiende que $nJ$ es $FK$ de $V$.\\ 
	Definimos:
	$$\rho_{calle\rightarrow c}(V) \land \rho_{NombEmp\rightarrow neJ}(J) \land \rho_{NombJefe \rightarrow nJ}(J)$$
	Luego, obtenemos la direccion de los jefes:
	$$\gamma \leftarrow \amalg_{nJ}(V \innerjoin[neV = nJ] J)$$
	La relacion $\gamma (\underline{neJ}, \underline{c}, \underline{cv})$ tal que $c$ y $cv$ son las calles y ciudades - respectivamente - de los jefes de $neJ$.\\
	La operacion que nos hace obtener aquellos empleados cuyas direcciones concuerdan con \textbf{cada una} de las direcciones de sus jefes es:
	$$\Pi_{neV} (V \div \gamma)$$
	No obstante dado que un $``$Empleado$"$  puede tener varios domicilios y sabiendo que $nJ \in V$, la operacion anterior no nos es del todo valida porque en la consulta se pide por la misma direccion que la de $``$su$"$ jefe - sin indicar que jefe.\\
	En caso de especificar en la consulta que la direccion del empleado sea la misma que la de \textbf{al menos uno} de sus jefes, la consulta seria la siguiente:
	$$\Pi_{neV}(V\cap \gamma)$$
	
	% f)
	\item Definimos, con el fin de sintetizar posteriormente;
	$$\delta \leftarrow (\sigma_{ncS = ``IBM"} (S))$$
	Obtenemos las ciudades \textit{sede} de $``$IBM$"$:
	$$\lambda \leftarrow \Pi_{cs}(\delta)$$
	Asi pues, las empresas que tienen sede en cada una de las ciudades en las que tiene sede IBM son:
	$$\theta \leftarrow S \div \lambda$$
	Sin embargo, en la consulta anterior sale incluida la propia IBM. Por tanto, realizamos lo siguiente:
	$$\theta - \Pi_{ncS}(\delta)$$
	
\end{enumerate}
	
	\newpage
	
	
\end{document}