Con el fin de abreviar el modelo, definimos:
	\begin{center}
		\begin{tabular}{l}
			$P \equiv Proveedores(\underline{idp}, nombreP, categoria, ciudad)$\\
			$C \equiv Componentes(\underline{idc}, nombreC, color, peso, ciudad)$\\
			$A \equiv Articulos(\underline{ida}, nombreA, ciudad)$\\
			$E \equiv Envios(\underline{idp}, \underline{idc}, \underline{ida}, cantidad)$\\
		\end{tabular}
	\end{center}
	\begin{enumerate}
		% a)
		\item 
		$$\Pi_{idp}(\sigma_{idc = C1 \wedge ida = A1}(E))$$
		
		% b)
		\item 
		$$\Pi_{ida}(\sigma_{idp = P1}(E))$$
		
		% c)
		\item Definimos:
		$$\rho_{ciudad\rightarrow c_1}(A)$$
		Luego,
		$$\Pi_{idp}(\sigma_{color=``rojo"}(C\innerjoin[C.idc = E.idc](\sigma_{c_1 = ``Segovia" \vee c_1 = ``Barcelona"}(E \innerjoin[E.ida = A.ida] A))))$$\\[0.25cm]
		Recordemos que $w_1 \innerjoin w_2 \equiv w_2 \innerjoin w_1$ , por lo que realmente da igual filtrar primero por %
		componentes \textit{rojos} enviados; que por artículos enviados fabricados en \textit{Segovia} o \textit{Barcelona}.
		
		% d)
		\item Definimos:
		$$\rho_{ciudad\rightarrow c_2}(P)$$
		Luego,
		$$\Pi_{idc}(\sigma_{c_2 = ``Segovia"} (P \innerjoin[P.idp = E.idp] (\sigma_{c_1 = ``Segovia"} (E \innerjoin[E.ida = A.ida] A))))$$
		
		% e)
		\item $$\Pi_{color}(C\innerjoin[C.idc = E.idc](\sigma_{ipd = P1}(E)))$$
		
		% f)
		\item Definimos:
		$$\rho_{ciudad\rightarrow c_3}(C)$$
		Luego,
		$$E\innerjoin[E.ida = A.ida] (A\innerjoin[c_1 = c_2](P\innerjoin[c_2 = c_2]C))$$
	\end{enumerate}